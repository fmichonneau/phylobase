\documentclass{article}
%\VignettePackage{phylo4}
% \VignetteIndexEntry{phylo4: classes and methods for phylogenetic trees and data}
\usepackage{graphicx}
\usepackage[colorlinks=true,urlcolor=blue]{hyperref}

\usepackage{color}

\usepackage[utf8]{inputenc} % for UTF-8/single quotes from sQuote()
\newcommand{\code}[1]{{{\tt #1}}}
\title{The \code{phylo4} S4 classes and methods}
\author{Ben Bolker \& Peter Cowan}
\date{\today}
\usepackage{/Library/Frameworks/R.framework/Resources/share/texmf/Sweave}
\begin{document}
\maketitle
\tableofcontents

\section{Introduction}

This document describes the new \code{phylo4} S4 classes and methods, which are intended to provide a unifying standard for phylogenetic data in R.  The base \code{phylo4} class is modeled on the the \code{phylo} class in \code{ape}.  \code{phylo4d} and \code{multiphylo4} extend the \code{phylo4} class to include data or multiple trees respectively.  In addition to describing the classes and methods this vignette gives examples of how they might be used.

Two motivations for the development of this package were better data checking and abstraction of the tree data formats.  Currently \code{phylobase} is capable of checking that data and trees are associated in the proper fashion, and protects users and developers from accidently reordering one, but not the other.  The \code{phylobase} package also seeks to abstract the data format so that commonly desired attributes can be accessed without knowing the underlying data structure.  This is achieved through generic functions which should ease development and protect functions from possible future changes in the data format.

\section{Package Overview}

The phylobase package currently implements functions and data structures that implement the following:

\begin{itemize}
  \item Data structures for storing a tree and multiple trees
  \item A data structure for storing a tree with associated tip and node data
  \item A data structure for storing multiple trees with one set of tip data
  \item Functions for reading nexus files into the above data structures
  \item Functions for converting between the above data structures and \code{ape phylo} objects as well as \code{ade4 phylog} objects
  \item Functions for subsetting, replacing, and plotting the above structures
\end{itemize}

\section{Using the S4 help system}

The \code{S4} help system works similarly to the \code{S3} help system with some small differences relating to how \code{S4} methods are written.  The \code{plot()} function is a good example.  When we type \code{?plot} we are provided the help for the default plotting function which expects \code{x} and \code{y}.  \code{R} also provides a way to smartly dispatch the right type of plotting function.  In the case of an \code{ape phylo} object (a \code{S3} class object) \code{R} evaluates the class of the object and finds the correct functions, so the following works correctly.

\begin{Schunk}
\begin{Sinput}
> library(ape)
> rand_tree <- rcoal(10)
> plot(rand_tree)
\end{Sinput}
\end{Schunk}

However, typing \code{?plot} still takes us to the default \code{plot} help.  We have to type \code{plot.phylo} to find what we are looking for.  This is because \code{S3} generics are simply functions with a dot and the class name added.  

The \code{S4} generic system is too complicated to describe here, but doesn't include the same dot notation.  As a result \code{?plot.phylo4} doesn't work, \code{R} does, however, find the right plotting function.

\begin{Schunk}
\begin{Sinput}
> library(phylobase)
> rand_p4_tree <- as(rand_tree, "phylo4")
> plot(rand_p4_tree)
\end{Sinput}
\end{Schunk}

All fine and good, but how to we find out about all the great features of the \code{phylobase} plotting function?  \code{R} has two nifty ways to find it, the first is to simply put a question mark in front of the whole call:

\begin{verbatim}
	> ?plot(rand_p4_tree)
\end{verbatim}

\code{R} looks at the class of the \code{rand\_p4\_tree} object and takes us to the correct help file (note: this only works with \code{S4} objects).  The second ways is handy if you already know the class of your object, or want to compare to generics for different classes:

\begin{verbatim}
	> method?plot("phylo4")
\end{verbatim}

More information about how \code{S4} documentation works can be found in the methods package, but running the following command.

\begin{Schunk}
\begin{Sinput}
> help("Documentation", package = "methods")
\end{Sinput}
\end{Schunk}

\section{Trees without data}

You can start with a tree --- an object of
class \code{phylo} from the \code{ape} package
(e.g., read in using the \code{read.tree()} or \code{read.nexus()}
functions), and convert it to a \code{phylo4} object.

For example, get the \emph{Geospiza} data from the \code{geiger} package:
\begin{Schunk}
\begin{Sinput}
> library(geiger)
> data(geospiza)
> names(geospiza)
\end{Sinput}
\begin{Soutput}
[1] "geospiza.tree" "geospiza.data"
\end{Soutput}
\end{Schunk}

Convert the \code{S3} tree to a \code{S4 phylo4} object using the \code{as()} function:
\begin{Schunk}
\begin{Sinput}
> library(phylobase)
> g1 <- as(geospiza$geospiza.tree, "phylo4")
> g1
\end{Sinput}
\begin{Soutput}
Phylogenetic tree with 14 tips and 13 internal nodes

Tip labels:
	fuliginosa, fortis, magnirostris, conirostris, scandens, difficilis, ...

Node labels:
	N01, N02, N03, N04, N05, N06, ...

Edge labels:
	E16, E17, E18, E19, E20, E21, ...

Rooted; includes branch lengths
\end{Soutput}
\end{Schunk}

Note that the nodes and edges are given default names if the tree contains no node or edge names.

The \code{summary} method gives a little extra information, including information on branch lengths:
\begin{Schunk}
\begin{Sinput}
> summary(g1)
\end{Sinput}
\begin{Soutput}
  No root edge.

 Phylogenetic tree : g1 

 Number of tips    : 14 
 Number of nodes   : 13 
 Branch lengths:
        mean         : 0.1764008 
        variance     : 0.04624379 
        distribution :
   Min. 1st Qu.  Median 3rd Qu.    Max. 
0.00917 0.04985 0.08000 0.21910 0.88080 
\end{Soutput}
\end{Schunk}

Print tip labels:
\begin{Schunk}
\begin{Sinput}
> labels(g1)
\end{Sinput}
\begin{Soutput}
 [1] "fuliginosa"   "fortis"       "magnirostris" "conirostris"  "scandens"    
 [6] "difficilis"   "pallida"      "parvulus"     "psittacula"   "pauper"      
[11] "Platyspiza"   "fusca"        "Pinaroloxias" "olivacea"    
\end{Soutput}
\end{Schunk}

Print internal node labels (R automatically assigns values):
\begin{Schunk}
\begin{Sinput}
> NodeLabels(g1)
\end{Sinput}
\begin{Soutput}
 [1] "N01" "N02" "N03" "N04" "N05" "N06" "N07" "N08" "N09" "N10" "N11" "N12"
[13] "N13"
\end{Soutput}
\end{Schunk}

Print edge labels (also automatically assigned):
\begin{Schunk}
\begin{Sinput}
> EdgeLabels(g1)
\end{Sinput}
\begin{Soutput}
 [1] "E16" "E17" "E18" "E19" "E20" "E21" "E22" "E23" "E24" "E1"  "E2"  "E3" 
[13] "E4"  "E5"  "E6"  "E25" "E7"  "E26" "E27" "E8"  "E9"  "E10" "E11" "E12"
[25] "E13" "E14"
\end{Soutput}
\end{Schunk}

Is it rooted?
\begin{Schunk}
\begin{Sinput}
> isRooted(g1)
\end{Sinput}
\begin{Soutput}
[1] TRUE
\end{Soutput}
\end{Schunk}

Which node is the root?
\begin{Schunk}
\begin{Sinput}
> rootNode(g1)
\end{Sinput}
\begin{Soutput}
[1] 15
\end{Soutput}
\end{Schunk}

Does it have any polytomies?
\begin{Schunk}
\begin{Sinput}
> hasPoly(g1)
\end{Sinput}
\begin{Soutput}
[1] FALSE
\end{Soutput}
\end{Schunk}

Does it have branch lengths?
\begin{Schunk}
\begin{Sinput}
> hasEdgeLength(g1)
\end{Sinput}
\begin{Soutput}
[1] TRUE
\end{Soutput}
\end{Schunk}

You can modify labels and other aspects
of the tree --- for example,
\begin{Schunk}
\begin{Sinput}
> labels(g1) <- tolower(labels(g1))
\end{Sinput}
\end{Schunk}

\section{Trees with data}

The \code{phylo4d} class matches trees with data.
(\textbf{fixme: need to be able to use ioNCL!})
or combine it with a data frame to make a \code{phylo4d} (tree-with-data)
object.

Now we'll take the \emph{Geospiza} data from \verb+geospiza$geospiza.data+
and merge it with the tree.  However, since \emph{G. olivacea} is included
in the tree but not in the data set, we will initially run into some trouble:

\begin{Schunk}
\begin{Sinput}
> g2 <- phylo4d(g1, geospiza$geospiza.data)
\end{Sinput}
\end{Schunk}

gives
\begin{Schunk}
\begin{Soutput}
Error in check_data(res, ...) : 
  Tip data names are a subset of tree tip labels 
(missing data names: platyspiza,pinaroloxias,olivacea) 
(extra data names: Pinaroloxias,Platyspiza)
\end{Soutput}
\end{Schunk}

We have two problems --- the first is that we forgot to lowercase
the labels on the data to match the tip labels:

\begin{Schunk}
\begin{Sinput}
> gdata <- geospiza$geospiza.data
> row.names(gdata) <- tolower(row.names(gdata))
\end{Sinput}
\end{Schunk}

To deal with the second problem
(missing data for \emph{G. olivacea}), we have a few choices.
The easiest is to use \code{missing.tip.data="OK"}
to allow R to create the new object:
\begin{Schunk}
\begin{Sinput}
> g2 <- phylo4d(g1, gdata, missing.tip.data = "OK")
\end{Sinput}
\end{Schunk}
(setting \code{missing.tip.data} to \code{"warn"}
would create the new object but print a warning).

Another way to deal with this would be to 
use \code{prune()} to drop
the offending tip from the tree first:
\begin{Schunk}
\begin{Sinput}
> g1B <- prune(g1, "olivacea")
> phylo4d(g1B, gdata)
\end{Sinput}
\end{Schunk}

You can summarize the new object:
\begin{Schunk}
\begin{Sinput}
> summary(g2)
\end{Sinput}
\begin{Soutput}
  No root edge.

 Phylogenetic tree : as(object, "phylo4") 

 Number of tips    : 14 
 Number of nodes   : 13 
 Branch lengths:
        mean         : 0.1764008 
        variance     : 0.04624379 
        distribution :
   Min. 1st Qu.  Median 3rd Qu.    Max. 
0.00917 0.04985 0.08000 0.21910 0.88080 

Comparative data:

Tips: data.frame with 14 taxa and 5 variables 

     wingL          tarsusL         culmenL          beakD      
 Min.   :3.975   Min.   :2.807   Min.   :1.974   Min.   :1.191  
 1st Qu.:4.189   1st Qu.:2.929   1st Qu.:2.187   1st Qu.:1.941  
 Median :4.235   Median :2.980   Median :2.311   Median :2.073  
 Mean   :4.236   Mean   :2.991   Mean   :2.333   Mean   :2.083  
 3rd Qu.:4.265   3rd Qu.:3.039   3rd Qu.:2.430   3rd Qu.:2.347  
 Max.   :4.420   Max.   :3.271   Max.   :2.725   Max.   :2.824  
 NA's   :1.000   NA's   :1.000   NA's   :1.000   NA's   :1.000  
     gonysW     
 Min.   :1.401  
 1st Qu.:1.845  
 Median :1.962  
 Mean   :2.014  
 3rd Qu.:2.222  
 Max.   :2.676  
 NA's   :1.000  

Object contains no node data.
\end{Soutput}
\end{Schunk}

Or use \code{tdata()} to extract the data (i.e., \code{tdata(g2)}). By default, \code{tdata()} will retrieve tip data, but you can also get internal node data only (\code{tdata(tree,"node")}) or --- if the tip and node data have the same format --- all the data combined (\code{tdata(tree,"allnode")}).

Plotting calls \code{plot.phylog} from the \code{ade4} package.

If you want to plot the data (e.g. for checking the input), \code{plot(tdata(g2))} will create the default plot for the data --- in this case, since it is a data frame [\textbf{this may change in future versions but should remain transparent}] this will be a \code{pairs} plot of the data.

\section{Subsetting}

The \code{subset} command offers a variety of ways of extracting portions of a \code{phylo4} or \code{phylo4d} tree, keeping any tip/node data consistent.

\begin{description}
\item[tips.include]{give a vector of tips (names or numbers) to retain}
\item[tips.exclude]{give a vector of tips (names or numbers) to drop}
\item[mrca]{give a vector of node or tip names or numbers; extract the clade containing these taxa}
\item[node.subtree]{give a node (name or number); extract the subtree starting from this node}
\end{description}

Different ways to extract the \emph{fuliginosa}-\emph{scandens}
clade:
\begin{Schunk}
\begin{Sinput}
> subset(g2, tips.include = c("fuliginosa", "fortis", "magnirostris", 
+     "conirostris", "scandens"))
> subset(g2, node.subtree = "N07")
> subset(g2, mrca = c("scandens", "fortis"))
\end{Sinput}
\end{Schunk}

One could drop the clade by  doing
\begin{Schunk}
\begin{Sinput}
> subset(g2, tips.exclude = c("fuliginosa", "fortis", "magnirostris", 
+     "conirostris", "scandens"))
> subset(g2, tips.exclude = allDescend(g2, MRCA(g2, c("difficilis", 
+     "fortis"))))
\end{Sinput}
\end{Schunk}

Another approach is to pick the subtree graphically, by plotting the tree and using \code{identify}, which returns the identify of the node you click on with the mouse.

\begin{Schunk}
\begin{Sinput}
> plot(g1)
> n1 <- identify(g1)
> subset(g2, node.subtree = n1)
\end{Sinput}
\end{Schunk}

\section{Tree-walking}

\code{getDescend}, \code{getAncest},
\code{allDescend}, \code{allAncest},
\code{getNodeByLabel}, \code{getLabelByNode}:
not sure about the names or functionality of these (how much do we work in terms of labels and how much in terms of internal numbers?)

\section{multiPhylo classes}


%%%%%%%%%%%%%%%%%%%%%%%%%%%%%%%%%%%%%%%%%%%%%%%%%%%%%%%%%%%%%%%%%%%%%%%%%
%%% Appendices %%%%%%%%%%%%%%%%%%%%%%%%%%%%%%%%%%%%%%%%%%%%%%%%%%%%%%%%%%
%%%%%%%%%%%%%%%%%%%%%%%%%%%%%%%%%%%%%%%%%%%%%%%%%%%%%%%%%%%%%%%%%%%%%%%%%

\appendix
\section{Definitions/slots}

\subsection{phylo4}
Like \code{phylo}, the main components of
the \code{phylo4} class are:
\begin{description}
\item[edge]{an $N \times 2$ matrix of integers,
  where the first column \ldots}
\item[edge.length]{numeric list of edge lengths
(length $N$ or empty)}
\item[Nnode]{integer, number of nodes}
\item[tip.label]{character vector of tip labels (required)}
\item[node.label]{character vector of node labels (maybe empty)}
\item[root.edge]{integer defining root edge (maybe NA)}
\end{description}

We have defined basic methods for \code{phylo4}:\code{show}, \code{print} (copied from \code{print.phylo} in\code{ape}), and a variety of accessor functions (see help files). \code{summary} does not seem to be terribly useful in the context of a ``raw'' tree, because there is not much to compute: \textbf{end users?}

Print method: add information about (ultrametric, scaled, polytomies (zero-length or structural))?

\subsection{phylo4d}

The \code{phylo4d} class extends \code{phylo4} with data.  Tip data, (internal) node data, and edge data are stored separately, but can be retrieved together or separately with \code{tdata(x,"tip")} or \code{tdata(x,"all")}.

\textbf{edge data can also be included --- is this
useful/worth keeping?}

\subsection{multiphylo4}

\section{Validity checking}

\begin{itemize}
\item number of rows of edge matrix ($N$) == length of edge-length vector (if $>0$)
\item (number of tip labels)+(nNode)-1 == $N$
\item data matrix must have row names
\item row names must match tip labels (if not, spit out mismatches)
\item 
\end{itemize}
 
Default node labels:

\section{Hacks/backward compatibility}

There is a way to hack the \verb+$+ operator so that it would provide backward compatibility with code that is extracting internal elements of a \code{phylo4}. The basic recipe is: 

\begin{Schunk}
\begin{Sinput}
> setMethod("$", "phylo4", function(x, name) {
+     attr(x, name)
+ })
\end{Sinput}
\end{Schunk}

but this has to be hacked slightly to intercept calls to elements that might be missing.  For example, \code{ape} detects whether log-likelihood, root edges, node labels, etc. are missing by testing whether they are \code{NULL}, whereas missing items are represented in \code{phylo4} by zero-length vectors in the slots (or \code{NA} for the root edge) --- so we need code like 
\begin{Schunk}
\begin{Sinput}
> if (!hasNodeLabels(x)) NULL else x@node.label
\end{Sinput}
\end{Schunk}
to handle these cases.


\end{document}
